%!TEX root = ../template.tex
%%%%%%%%%%%%%%%%%%%%%%%%%%%%%%%%%%%%%%%%%%%%%%%%%%%%%%%%%%%%%%%%%%%
%% chapter1.tex
%% NOVA thesis document file
%%
%% Chapter with introduction
%%%%%%%%%%%%%%%%%%%%%%%%%%%%%%%%%%%%%%%%%%%%%%%%%%%%%%%%%%%%%%%%%%%

\typeout{NT FILE chapter1.tex}%

\chapter{Introduction}
\label{cha:introduction}

% Esta secção introduz a motivação da tese, baseada na lacuna
% entre o potencial das APIs 5G e a sua aplicação prática.
\section{Motivation}
\label{sec:motivation}

The emergence of 5G networks represents a paradigm shift that extends far beyond simple improvements in speed. The true revolution lies in powerful new network capabilities, such as low latency and slicing, which allow users to not only request information from the network but also configure it dynamically for the first time. This evolution creates an unprecedented opportunity for telecommunications operators to expand their service offerings by transforming their networks into service enablement platforms.

Despite the undeniable potential of 5G, a significant challenge is delaying its realization: the intrinsic complexity of telecommunications networks. While powerful, 5G capabilities are technically complex. The vast majority of application developers lack the specific technical knowledge required to take advantage of 5G; therefore, an abstraction process is fundamental to enable innovation on top of the network. Abstraction transforms complex network functions into easy-to-consume, user-friendly Service APIs.

It is in this context that global initiatives such as Project CAMARA have emerged. An open source project under the Linux Foundation, in collaboration with the GSMA, CAMARA focuses on defining, developing, and testing a set of APIs. Its goal is to create a standard that ensures application portability and accelerates technological development through reference implementations and developer-friendly documentation.

MEO, having joined the project and by exposing a portfolio of CAMARA compliant APIs, positions itself at the forefront of this transformation.

However, the mere existence of these APIs does not guarantee adoption, nor does it automatically unlock the value of the 5G network. There is a significant gap between the theoretical potential of these tools and their practical application in innovative market solutions. Developers and companies need to understand, through concrete examples, how these APIs can solve real-world problems and create new experiences.

It is precisely this gap that motivates the present dissertation. This work is driven by the need to explore innovative and relevant use cases that demonstrate the tangible value of open network APIs within MEO’s specific 5G ecosystem. By characterizing these practical applications in detail, this research aims to validate the potential of these APIs, demonstrating that application-network integration is the key to materializing the promise of 5G.

% Esta secção define os objetivos da tese.
\section{Objectives}
\label{sec:objectives}

% Adicione aqui o texto dos seus objetivos.
% 
% Exemplo:
% O objetivo principal desta dissertação é explorar e validar casos de uso inovadores 
% para as APIs de rede abertas (CAMARA) no ecossistema 5G da MEO.
% 
% Para alcançar este objetivo, definem-se os seguintes objetivos específicos:
% \begin{itemize}
%     \item Caracterizar o estado-da-arte das iniciativas de APIs de rede, com foco no projeto CAMARA.
%     \item Analisar o portefólio de APIs disponibilizado pela MEO, detalhando a sua funcionalidade técnica.
%     \item Identificar e propor casos de uso de elevado valor acrescentado para as APIs...
%     \item ...
% \end{itemize}


% Esta secção descreve como o documento está organizado.
\section{Document Structure}
\label{sec:doc_stuct}

% Adicione aqui o texto da organização do documento.
% 
% Exemplo:
% Este documento encontra-se estruturado em 4 capítulos, que se descrevem de seguida.
% 
% \textbf{Capítulo \ref{cha:introduction} - Introdução:} Apresenta a motivação do tema, 
% os objetivos da dissertação e a estrutura geral do documento.
% 
% \textbf{Capítulo \ref{cha:fundamentals} - Fundamentação e Estado-da-Arte:} Descreve os 
% conceitos fundamentais sobre... (incluir aqui a referência ao CAMARA)
% 
% \textbf{Capítulo \ref{cha:planning} - Planeamento:} (Se aplicável) Detalha o 
% planeamento e metodologia seguidos...
% 
% \textbf{Capítulo \ref{cha:conclusions} - Conclusões:} Apresenta as conclusões 
% retiradas do trabalho desenvolvido, bem como sugestões para trabalho futuro.
%
% (Nota: Terá de ajustar os \label{} dos capítulos seguintes, ex: \ref{cha:fundamentals})